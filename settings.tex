%Dokumentenklasse "scrbook" - Erweitert um den Verweis auf die Verzeichnisse und Texteigenschaften
\documentclass[chapterprefix=true, 12pt, a4paper, oneside, parskip=half, listof=totoc, bibliography=totoc]{scrbook}

%Tweaks für scrbook
\usepackage{scrhack}

%Blindtext
\usepackage{blindtext}

%Stichwortverzeichnis
\usepackage{imakeidx}

%Glossar, Stichworverzeichnis
\usepackage[toc, acronym]{glossaries} % Akronyme werden als eigene Liste aufgeführt

%Anpassung von Kopf- und Fußzeile
\usepackage[automark,headsepline]{scrlayer-scrpage}
\automark{chapter}
\ihead{\leftmark}
\chead{}
\ohead{\thepage}
\ifoot*{}
\cfoot[\thepage]{}
\ofoot*{}
\pagestyle{scrheadings}

%Auskommentieren für die Verkleinerung des vertikalen Abstandes eines neuen Kapitels
%\renewcommand*{\chapterheadstartvskip}{\vspace*{.25\baselineskip}}

%Anpassung der Seitenränder ermöglichen
\usepackage{geometry}

%Zeilenabstand 1,5
\usepackage[onehalfspacing]{setspace}

%Verbesserte Darstellung der Buchstaben zueinander
\usepackage[stretch=10]{microtype}

%Deutsche Bezeichnungen für angezeigte Namen (z.B. Innhaltsverzeichnis etc.)
\usepackage[ngerman]{babel}

%Unterstützung von Umlauten und anderen Sonderzeichen (UTF-8)
\usepackage{lmodern}
\usepackage[utf8]{luainputenc}
\usepackage[T1]{fontenc}

%Verwendung von Akronymen
\usepackage[printonlyused]{acronym}

%Unterstützung der H positionierung (keine automatische Verschiebung eingefügter Elemente)
\usepackage{float} 

%Erlaubt Umbrüche innerhalb von Tabellen
\usepackage{tabularx}

%Erlaubt die Darstellung von Sourcecode mit Highlighting
\usepackage{listings}

%Definierung eigener Farben bei nutzung eines selbst vergebene Namens
\usepackage{xcolor}

%Vektorgrafiken
\usepackage{tikz}

%Grafiken (wie jpg, png, etc.)
\usepackage{graphicx}

%Ermöglicht Verknüpfungen innerhalb des Dokumentes (e.g. for PDF), Links werden durch "hidelink" nicht explizit hervorgehoben
\usepackage[hidelinks,german]{hyperref}

%Einbindung und Verwaltung von Literaturverzeichnissen
\usepackage{csquotes} %wird von biber benötigt
\usepackage[style=alphabetic, backend=biber, bibencoding=ascii]{biblatex}
\addbibresource{references/references.bib}

%---------------------------------------------------------------------------%

%Anpassung der Überschriften
\addtokomafont{disposition}{\rmfamily}

%Zusätzliche Farben
\definecolor{darkgreen}{rgb}{0,0.9,0}
\definecolor{pblue}{rgb}{0.13,0.13,1}
\definecolor{pLightblue}{HTML}{3E7EFF} %auch in RGB-Darstellung
\definecolor{pgreen}{rgb}{0,0.5,0}
\definecolor{pred}{rgb}{0.9,0,0}
\definecolor{pgrey}{rgb}{0.46,0.45,0.48}

%Umbenennungen
\renewcommand{\lstlistlistingname}{Quelltextverzeichnis}

%Anpassugen zur Quelltextdarstellung, kann bei Bedarf überschrieben werden (z.B. wenn unterschiedliche Sprachen zum Einsatz kommen)
\renewcommand{\lstlistingname}{Codeauszug}
\lstset{
	language=Java,
	columns=fullflexible,
	aboveskip=5pt,
	belowskip=10pt,
	basicstyle=\small\ttfamily,
	backgroundcolor=\color{black!5},
	commentstyle=\color{pgreen},
	keywordstyle=\color{pblue},
	stringstyle=\color{pred},
	showspaces=false,
	showstringspaces=false,
	showtabs=false,
	xleftmargin=10pt,
	xrightmargin=0pt,
	framesep=5pt,
	framerule=3pt,
	frame=leftline,
	rulecolor=\color{darkgreen},
	tabsize=2,
	breaklines=true,
	breakatwhitespace=true,
	prebreak={\mbox{$\hookleftarrow$}}
}